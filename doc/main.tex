\documentclass[a4paper,10pt]{llncs}

% Set margin
\usepackage[hmargin=1in,vmargin=1in]{geometry}

% To support table of contents in llncs.
\setcounter{tocdepth}{3}
\setcounter{secnumdepth}{4}

\makeatletter
\renewcommand*\l@author[2]{}
\renewcommand*\l@title[2]{}
\makeatother

\pagestyle{plain}

%% CONFIG

\def\isanonymous{0}
\def\buildexternal{1}

%% There is a package that warns you about
%% obsolete commands and package. https://daniel-j-h.github.io/post/latex-a-modern-approach/
\usepackage[l2tabu,orthodox]{nag}

%% MISC PACKAGES

\usepackage{microtype}
\usepackage[bookmarksdepth=2, backref=page]{hyperref}
\usepackage{booktabs}  %% tables\part{title}
\usepackage{comment}
\usepackage{enumitem}
\usepackage{multicol}
\usepackage{amsmath,amssymb}
\usepackage{bbm}
\usepackage{ragged2e}
\usepackage{adjustbox}
\usepackage{empheq}

\usepackage{etoolbox}
\usepackage{graphicx}
\graphicspath{ {./images/} }

%% TODO NOTES

\usepackage[dvipsnames]{xcolor}
\definecolor{oxygenorange}{HTML}{FFDD00}
\usepackage{todonotes}
\usepackage{xargs}
\newcommandx{\unsure}[2][1=]{\todo[linecolor=red,backgroundcolor=red!25,bordercolor=red,#1]{#2}}
\newcommandx{\change}[2][1=]{\todo[linecolor=blue,backgroundcolor=blue!25,bordercolor=blue,#1]{#2}}
\newcommandx{\info}[2][1=]{\todo[linecolor=OliveGreen,backgroundcolor=OliveGreen!25,bordercolor=OliveGreen,#1]{#2}}
\newcommandx{\improvement}[2][1=]{\todo[linecolor=Plum,backgroundcolor=Plum!25,bordercolor=Plum,#1]{#2}}
\newcommandx{\thiswillnotshow}[2][1=]{\todo[disable,#1]{#2}}

\newcommand{\binyi}[2][inline,color=oxygenorange]{\todo[#1]{\textbf{binyi:} #2}\xspace}
\newcommand{\philippe}[2][inline,color=oxygenorange]{\todo[#1]{\textbf{philippe:} #2}\xspace}
\newcommand{\alex}[2][inline,color=oxygenorange]{\todo[#1]{\textbf{alex:} #2}\xspace}
\newcommand{\fernando}[2][inline,color=oxygenorange]{\todo[#1]{\textbf{Fernando:} #2}\xspace}

%% ANONYMOUS SUBMISSIONS

\usepackage{ifthen}
\newcommand{\anonymous}[2]{%
\ifthenelse{\equal{\isanonymous}{1}}%
{{#1}}%
{{#2}}%
}%

%% SYMBOLS & NOTATION

\usepackage{amsmath,amsfonts,amssymb}
\usepackage{xspace}
% \usepackage[advantage,adversary,lambda,landau,operators,probability,sets,logic,complexity,asymptotics]{cryptocode}
\usepackage{seqsplit}

\renewcommand{\vec}[1]{\ensuremath{\boldsymbol{#1}}\xspace}
\newcommand{\mat}[1]{\ensuremath{\boldsymbol{#1}}\xspace}

%% PLOTS

\usepackage{tikz,pgfplots}
\pgfplotsset{compat=newest}
\usetikzlibrary{calc}
\usetikzlibrary{arrows}

\usetikzlibrary{external}
\tikzexternalize[prefix=plots/]
\tikzset{external/optimize=false}
\tikzset{external/export=false}

\newcommand{\tikzexternalizemaybe}{%
  \ifthenelse{\equal{\buildexternal}{1}}%
  {\tikzset{external/export=true}}%
  {\tikzset{external/export=false}%
}}%

% from pgfplotsthemetol.sty
\definecolor{DarkPurple}{HTML}{332288}
\definecolor{DarkBlue}{HTML}{6699CC}
\definecolor{LightBlue}{HTML}{88CCEE}
\definecolor{DarkGreen}{HTML}{117733}
\definecolor{DarkRed}{HTML}{661100}
\definecolor{LightRed}{HTML}{CC6677}
\definecolor{LightPink}{HTML}{AA4466}
\definecolor{DarkPink}{HTML}{882255}
\definecolor{LightPurple}{HTML}{AA4499}
\definecolor{DarkBrown}{HTML}{604c38}
\definecolor{DarkTeal}{HTML}{23373b}
\definecolor{LightBrown}{HTML}{EB811B}
\definecolor{LightGreen}{HTML}{14B03D}
\definecolor{NewMidnightBlue}{HTML}{000099}

\pgfplotsset{width=1.0\textwidth,
  height=0.45\textwidth,
  cycle list={%
    solid,\\%
    dotted,DarkBlue,very thick\\%
    dashed,DarkGreen,thick\\%
    loosely dotted,very thick,DarkRed\\%
    loosely dashed,black,very thick,DarkBrown\\%
    loosely dashdotted,darkgray,very thick,DarkTeal\\%
    \\%
  },
  legend pos=north west,
  legend style={fill=none},
  legend cell align={left}}

%% LISTINGS

\usepackage{listings}
\lstdefinelanguage{Sage}[]{Python}{morekeywords={True,False,sage,cdef,cpdef,ctypedef,self},sensitive=true}
\lstset{frame=none,
          showtabs=False,
          showspaces=False,
          showstringspaces=False,
          commentstyle=\color{gray!80!black},
          keywordstyle={\color{gray!80!black}\textbf},
          stringstyle ={\color{gray!80!black}},
          basicstyle=\tt\small\relax,,
        }

\usepackage{navigator}

% HANDLING OF CODE
\usepackage{algorithm}
\usepackage[advantage,adversary,lambda,landau,operators,probability,sets,logic,complexity,asymptotics]{cryptocode}


%% PDF SETUP

\hypersetup{colorlinks=true,citecolor=NewMidnightBlue,linkcolor=NewMidnightBlue}

%% Enumitem SETUP

\setitemize[1]{label=$\bullet$}
\setitemize[2]{label=$\circ$}
\setitemize[3]{label=$\star$}

%%%% these patches ensure that the backrefs point to the actual occurrences of the citations in the text, not just the page or section in which they appeared
%%%% https://tex.stackexchange.com/questions/54541/precise-back-reference-target-with-hyperref-and-backref
%%%% BEGIN BACKREF DIRECT PATCH, apply these AFTER loading hyperref package with appropriate backref option
% The following options are provided for the patch, currently with a poor interface!
% * If there are multiple cites on the same (page|section) (depending on backref mode),
%   should we show only the first one or should we show them all?
\newif\ifbackrefshowonlyfirst
\backrefshowonlyfirstfalse
%\backrefshowonlyfirsttrue
%%%% end of options
%
% hyperref is essential for this patch to make any sense, so it is not unreasonable to request it be loaded before applying the patch
\makeatletter
% 1. insert a phantomsection before every cite, so hyperref has something to target
%    * in case natbib is loaded. hyperref provides an appropriate hook so this should be safe, and we don't even need to check if natbib is loaded!
\let\BR@direct@old@hyper@natlinkstart\hyper@natlinkstart
\renewcommand*{\hyper@natlinkstart}{\phantomsection\BR@direct@old@hyper@natlinkstart}% note that the anchor will appear after any brackets at the start of the citation, but that's not really a big issue?
%    * if natbib isn't used, backref lets \@citex to \BR@citex during \AtBeginDocument
%      so just patch \BR@citex
\let\BR@direct@oldBR@citex\BR@citex
\renewcommand*{\BR@citex}{\phantomsection\BR@direct@oldBR@citex}%

% 2. if using page numbers, show the page number but still hyperlink to the phantomsection instead of just the page!
\long\def\hyper@page@BR@direct@ref#1#2#3{\hyperlink{#3}{#1}}

% check which package option the user loaded (pages (hyperpageref) or sections (hyperref)?)
\ifx\backrefxxx\hyper@page@backref
    % they wanted pages! make sure they get our re-definition
    \let\backrefxxx\hyper@page@BR@direct@ref
    \ifbackrefshowonlyfirst
        %\let\backrefxxxdupe\hyper@page@backref% test only the page number
        \newcommand*{\backrefxxxdupe}[3]{#1}% test only the page number
    \fi
\else
    \ifbackrefshowonlyfirst
        \newcommand*{\backrefxxxdupe}[3]{#2}% test only the section name
    \fi
\fi

% 3. now make sure that even if there is no numbered section, the hyperref's still work instead of going to the start of the document!
\RequirePackage{etoolbox}
\patchcmd{\Hy@backout}{Doc-Start}{\@currentHref}{}{\errmessage{I can't seem to patch backref}}
\makeatother
%%%% END BACKREF PATCHES

\raggedbottom

%% OPENING

\title{Hyperplonk}
\author{}
\institute{}

\begin{document}

\setlength{\parskip}{0pt}

\maketitle
% Maths
\newcommand{\unisample}{\xleftarrow{\$}\xspace}
\newcommand{\assign}{\ensuremath{\leftarrow}\xspace}
\newcommand{\relation}{\mathcal{R}}
\newcommand{\isnegl}{\leq \mu(\secpar)}
\newcommand{\zo}{\set{0,1}}
\newcommand{\universe}{\mathcal{U}}
\newcommand{\condpr}[2]{
	\Pr \left[\:{#1}\, \middle|\, {#2}\:\right]
} %conditional probability
\newcommand{\iseq}{\stackrel{?}{=}}

\newcommand{\calA}{\mathcal{A}}
\newcommand{\calB}{\mathcal{B}}
\newcommand{\calC}{\mathcal{C}}
\newcommand{\calE}{\mathcal{E}}
\newcommand{\calG}{\mathcal{G}}
\newcommand{\calH}{\mathcal{H}}
\newcommand{\calI}{\mathcal{I}}
\newcommand{\calK}{\mathcal{K}}
\newcommand{\calL}{\mathcal{L}}
\newcommand{\calM}{\mathcal{M}}
\newcommand{\calO}{\mathcal{O}}
\newcommand{\calP}{\mathcal{P}}
\newcommand{\calR}{\mathcal{R}}
\newcommand{\calS}{\mathcal{S}}
\newcommand{\calT}{\mathcal{T}}
\newcommand{\calU}{\mathcal{U}}
\newcommand{\calV}{\mathcal{V}}
\newcommand{\calX}{\mathcal{X}}
\newcommand{\calY}{\mathcal{Y}}
\mathchardef\mhyphen="2D

% general crypto
\newcommand{\deq}{\mathrel{\triangleq}}
\newcommand{\qeq}{\ensuremath{\stackrel{?}{=}}}
\newcommand{\ASp}{\mathbb{A}\!^1}
\newcommand{\C}{\mathcal{C}}
\newcommand{\gHH}{g_\mathbb{H}}
\newcommand{\range}[1]{[{#1}]}
\newcommand{\prot}[1]{\mathcal{{#1}}}
\newcommand{\PK}{\mathit{pk}}
\newcommand{\VK}{\mathit{vk}}
\newcommand{\SK}{\mathit{sk}}
\newcommand{\uid}{\mathit{uid}}
\newcommand{\ext}{\ensuremath{\mathcal{E}}\xspacemm}
\newcommand{\Rel}{\ensuremath{\mathfrak{R}}\xspacemm}
\newcommand{\RelI}{\ensuremath{\mathfrak{R}_I}\xspacemm}
\newcommand{\TT}{\mathcal{T}}

\newcommand{\Adversary}{\ensuremath{\mathcal{A}}}
\newcommand{\oracle}{\mathcal{O}}
\renewcommand{\pp}{\ensuremath{\mathsf{pp}}} % macro already defined in the `crypto` package
\renewcommand{\secpar}{\lambda} % macro already defined in the `crypto` package
\renewcommand{\FF}{\ensuremath{\mathbb{F}}}
\renewcommand{\NN}{\ensuremath{\mathbb{N}}}
\renewcommand{\GG}{\ensuremath{\mathbb{G}}}
\newcommand{\accept}{\ensuremath{\mathsf{accept}}\xspace}
\newcommand{\reject}{\ensuremath{\mathsf{reject}}\xspace}
\newcommand{\InputSpace}{\ensuremath{\mathcal{X}}}
\newcommand{\OutputSpace}{\ensuremath{\mathcal{Y}}}
\newcommand{\state}{\ensuremath{\mathsf{st}}}
\newcommand{\simulator}{\ensuremath{\mathcal{S}}}
\newcommand{\st}{\ensuremath{\mathsf{st}}\xspace}

%% Elliptic curves
\newcommand{\pbw}{\ensuremath{q}}
\newcommand{\pbls}{\ensuremath{p}}
\newcommand{\rbls}{\ensuremath{r}}
\newcommand{\rjub}{\ensuremath{\mathbb{J}}}
\newcommand{\pairing}[2]{\ensuremath{e(#1, #2)}}
\newcommand{\base}{\ensuremath{\mathtt{G}}}
\newcommand{\GH}{\ensuremath{\mathtt{GH}}} % Hash to Group
% encode/represent field element in #1 to a field element in #2
\newcommand{\encode}[2]{\ensuremath{\mathsf{encode}_{#1}^{#2}}}

% Vectors
\newcommand{\myvec}[1]{\ensuremath{\vec{#1}}\xspace}
\newcommand{\xv}{\myvec{x}}
\newcommand{\yv}{\myvec{y}}
\newcommand{\rv}{\myvec{r}}

% Typography
\newcommand{\boxedeq}[1]{\begin{empheq}[box={\fboxsep=6pt\fbox}]{align}#1 \nonumber \end{empheq}}

\section{Preliminaries}
TODO

\section{Building block protocols}
\newcommand{\bcube}[1]{B_{#1}}
\newcommand{\bdp}[2]{{\mathcal{F}}_{#1}^{(\le{#2})}}   % bounded degree polynomials
\newcommand{\perm}{\pi} % permutation
\newcommand{\addr}{\mathsf{addr}} % lookup addresses
\newcommand{\eqpoly}{eq}
\newcommand{\sidpoly}{s_{\text{id}}}
\newcommand{\spermpoly}{s_{\perm}}
\newcommand{\decode}[1]{[#1]}
\newcommand{\nxt}[1]{\mathtt{nxt}_{#1}}
\newcommand{\sps}[1]{k_{#1}} % sparseness index for the next polynomial

% Indexed relations
\newcommand{\idxR}{\mathbbm{i}} % relation index
\newcommand{\instR}{\mathbbm{x}} % relation instance
\newcommand{\witR}{\mathbbm{w}} % relation witness

% Building block relations
\newcommand{\RPERM}{{\mathcal{R}}_{\text{PERM}}}
\newcommand{\RZERO}{{\mathcal{R}}_{\text{ZERO}}}
\newcommand{\RSUM}{{\mathcal{R}}_{\text{SUM}}}
\newcommand{\RPROD}{{\mathcal{R}}_{\text{PROD}}}
\newcommand{\RPLONK}{{\mathcal{R}}_{\text{PLONK}}}
\newcommand{\RLOOKUP}{{\mathcal{R}}_{\text{LOOKUP}}}
\newcommand{\RPUB}{{\mathcal{R}}_{\text{PUB}}}

% Poly IOPs
\newcommand{\prover}{\mathcal{P}} 
\newcommand{\verifier}{\mathcal{V}}
\newcommand{\indexer}{\mathcal{I}}

% Sumchecks
\newcommand{\chal}[1]{\rho_{#1}} % random challenges


\subsection{Relations}
Let $\bcube{n} \deq \zo^{n} \subseteq \FF^n$ be the boolean hypercube. Let $\bdp{n}{d}$
be the set of multivariate polynomials in $\FF[X_1, \dots, X_n]$ where the degree 
in each variable is at most $d$. We define the following important relations.

\begin{definition}[Sumcheck relation]
    The relation $\RSUM$ is the set of all tuples $(\instR, \witR) = 
    ((n, d, s), f)$ where $f \in \bdp{n}{d}$ and $\sum_{\xv \in \bcube{n}} f(\xv) = s$.
\end{definition}

\begin{definition}[Zerocheck relation]
    The relation $\RZERO$ is the set of all tuples $(\instR, \witR) =
    ((n, d), f)$ where $f \in \bdp{n}{d}$ and $f(\xv) = 0$ for all $\xv \in \bcube{n}$.
\end{definition}

\begin{definition}[Product-check relation]
    The relation $\RPROD$ is the set of all tuples $(\instR, \witR) =
    ((n, d, s), f)$ where $f \in \bdp{n}{d}$ and $\prod_{\xv \in \bcube{n}} f(\xv) = s$.
\end{definition}

\begin{definition}[Permutation relation]
    The indexed relation $\RPERM$ is the set of triples $(\idxR, \instR, \witR)=
    ((n, d, \perm), \bot, (f, g))$ where $f, g \in \bdp{n}{d}$, $\perm : \bcube{n} \rightarrow \bcube{n}$
    is a permutation, and $g(\xv) = f(\perm(\xv))$ for all $\xv \in \bcube{n}$.
\end{definition}

\begin{definition}[Lookup relation]
    The indexed relation $\RLOOKUP$ is the set of triples $(\idxR, \instR, \witR)=
    ((n, d_t, d_f, t), \bot, (f, \addr))$, where $t \in \bdp{n}{d_t}$, $f \in \bdp{n}{d_f}$,
    and $\addr: \bcube{n} \rightarrow \bcube{n}$ is an address map such that 
    $f(\xv) = t(\addr(\xv))$ for all $\xv \in \bcube{n}$.
\end{definition}

\begin{remark}
    We can extend the lookup relation $\RLOOKUP$ to an \emph{online lookup relation}
    where the table polynomial $t(\xv)$ is part of the witness $\witR$ 
    instead of part of the index $\idxR$.
\end{remark}

\subsection{Constructions}
We specify the Polynomial Interactive Oracle Proofs (PIOPs) protocols for the above relations.

\subsubsection*{PIOP for $\RSUM$.} We construct a PIOP for 
the relation $\RSUM$. It is basically a reformulation to the sumcheck protocol
of~\cite{LFKN92}.
The prover $\prover$ takes as input an instance $\instR = (n, d, s)$,
and witness $\witR = f$; the verifier $\verifier$ takes as input the 
instance $\instR = (n, d, s)$.

\textbf{Interactive phase.} 
In the first round, the prover sends the (multi-variate) oracle $f$ to the verifier.
Let $\chal{i} \in \FF$ $(1 \le i \le n)$ denote the public coin challenge
of the verifier in round $i$.
For each round $i \in [n]$, the prover $\prover$ computes a 
degree-$d$ univariate polynomial
\[
      s_i(X_i) \deq 
      \sum_{ (x_{i+1}, \dots, x_{n}) \in \zo^{n-i} } 
      f(\chal{1}, \dots, \chal{i-1}, X_i, x_{i+1}, \dots, x_n) \,.
\]
and sends the (univariate) oracle $s_i$ to the verifier.

\textbf{Query phase.}
Define $s_0(\chal{0}) = s$. For each $i \in [n]$, the verifier makes oracle queries to $s_i(X_i)$ on points
$\{0, 1, \chal{i} \}$ and checks that
\[
  s_{i-1}(\chal{i-1}) = s_{i}(0) + s_{i}(1) \,.  
\]
The verifier further queries $f$ on point $(\chal{1}, \dots, \chal{n})$ and checks that
\[
  s_n(\chal{n}) = f(\chal{1}, \dots, \chal{n}).  
\]

\subsubsection*{PIOP for $\RZERO$.} 
We construct a PIOP for the relation $\RZERO$.
The prover $\prover$ takes as input an instance $\instR = (n, d)$,
and witness $\witR = f$; the verifier $\verifier$ takes as input the 
instance $\instR = (n, d)$. We define equality checking polynomial
\[
  \eqpoly(\xv, \yv) \deq \prod_{i=1}^{n} (x_iy_i + (1-x_i)(1-y_i)) \,,
\]
and observe that $(\instR, \witR) = ((n,d), f) \in \RZERO$ if and only if
\[
  g(\yv) \deq \sum_{\xv \in \bcube{n}} f(\xv) \cdot \eqpoly(\xv, \yv)  
\]
is identically zero. Hence it is sufficient to check that $g(\rv) = 0$
for a random vector $\rv \in \FF^n$. The PIOP is as follows:
\begin{itemize}
    \item The prover $\prover$ sends the oracle $f$ to $\verifier$.
    \item The verifier $\verifier$ sends $\prover$ a random vector $\rv \in \FF^{n}$.
    \item The prover $\prover$ and the verifier $\verifier$ run one PIOP execution
    for $\RSUM$ with $\instR = (n, d+1, 0)$ and $\witR = \hat{f}$, where
    \[
      \hat{f}(\xv) \deq f(\xv) \cdot \eqpoly(\xv, \rv) 
    \]
    is a polynomial in $\bdp{n}{d+1}$.
\end{itemize}

\subsubsection*{PIOP for $\RPROD$.}
We construct a PIOP for the relation $\RPROD$.
It is basically a reformulation to the protocol in Section 5
of~\cite{SL20}.
The prover $\prover$ takes as input an instance $\instR = (n, d, s)$
and a witness $\witR = f \in \bdp{n}{d}$; the verifier $\verifier$
takes as input the instance $\instR = (n, d, s)$.

\textbf{Interactive phase.} 
\begin{itemize}
    \item The prover $\prover$ computes a polynomial $\hat{f} \in \bdp{n+1}{d}$ such that 
    for all $\xv \in \bcube{n}$, it holds that
    \[
      \hat{f}(0, \xv) = f(\xv), \qquad \qquad \hat{f}(1, \xv) = \hat{f}(\xv, 0) \cdot \hat{f}(\xv, 1)\,.
    \]
    \item The prover $\prover$ sends the oracles $\hat{f}$, $f$ to $\verifier$.
    \item The prover $\prover$ and the verifier $\verifier$ runs the interactive phase of 
        the PIOP for $\RZERO$ with $\instR = (n, 2d)$ and $\witR = g \in \bdp{n}{2d}$ where
        \[
            g(\xv) \deq \hat{f}(1, \xv) - \hat{f}(\xv, 0) \cdot \hat{f}(\xv, 1) \,.
        \]
\end{itemize}

\textbf{Query phase.}
\begin{itemize}
    \item The verifier $\verifier$ queries $\hat{f}$ on point $(1, \dots, 1, 0)$ and checks that
        $\hat{f}(1,\dots, 1, 0) = s$.
    \item The verifier $\verifier$ queries $\hat{f}$ and $f$ on a random vector $\rv \in \FF^n$,
        and checks that $\hat{f}(0, \rv) = f(\rv)$.
    \item The verifier $\verifier$ runs the query phase of the PIOP 
    for $\RZERO$ with $\instR = (n, 2d)$ and $\witR = g$ (specified in the interactive phase).
\end{itemize}

\subsubsection*{PIOP for $\RPERM$.}
We construct a PIOP for the indexed relation $\RPERM$.
The construction makes use of the technique in~\cite{GWC19} that 
transforms a permutation relation to a product relation.
The indexer $\indexer$ takes as input an index $\idxR = (n, d, \perm)$;
the prover $\prover$ takes as input the index $\idxR = (n, d, \perm)$,
a witness $\witR = (f, g)$, and the ouput oracles from $\indexer$;
the verifier $\verifier$ takes as input the instance $\instR = \bot$ and 
the indexer oracles.

\textbf{Offline phase.}
\begin{itemize}
    \item The indexer computes two polynomials $\sidpoly, \spermpoly \in \bdp{n}{1}$ such that
    for all $\xv \in \bcube{n}$, it holds that 
    \[
        \sidpoly(\xv) = \decode{\xv}, \qquad\qquad \spermpoly(\xv) = \decode{\perm(\xv)}\,,
    \]
    where $\decode{\xv}$ denotes the decoding from a binary vector $\xv \in \bcube{n}$
    to a field element $y \in \FF$, that is, $\decode{\xv} = \sum_{i=1}^{n} \xv_i \cdot 2^{i-1}$.
    \item The indexer outputs the oracles $\sidpoly, \spermpoly$.
\end{itemize}

\textbf{Interactive phase.} 
\begin{itemize}
    \item The prover $\prover$ sends oracles $f, g$ to $\verifier$.
    \item The verifier $\verifier$ sends random challenges $\beta, \gamma \in \FF$.
    \item The prover $\prover$ and the verifier $\verifier$ run (in parallel) the interactive phase 
    of two $\RPROD$'s PIOP executions:
        \begin{itemize}
            \item the $1$st execution is with instance $\instR = (n, d, s)$ and
                witness $\witR = v_{f} \in \bdp{n}{d}$, where 
                $
                    s = \prod_{\xv\in\bcube{n}} (f(\xv) + \beta \sidpoly(\xv) + \gamma)
                $ and
                \[
                    v_{f}(\xv) \deq f(\xv) + \beta \sidpoly(\xv) + \gamma \,;
                \]
            \item the $2$nd execution is with instance $\instR = (n, d, s)$ and 
                witness $\witR = v_{g} \in \bdp{n}{d}$, where 
                \[
                    v_{g}(\xv) \deq g(\xv) + \beta \spermpoly(\xv) + \gamma \,.
                \]
        \end{itemize}
\end{itemize}

\textbf{Query phase.}
The verifier $\verifier$ runs (in parallel) the query phase of the two $\RPROD$ PIOP 
executions specified above.

\subsubsection*{PIOP for $\RLOOKUP$.} 
We construct a PIOP for the indexed relation $\RLOOKUP$.
The construction draws inspiration from the technique in~\cite{GW20} that 
transforms a lookup relation to a (special) product relation.
The indexer $\indexer$ takes as input an index $\idxR = (n, d_f, d_t, t)$;
the prover $\prover$ takes as input the index $\idxR$, instance $\instR = \bot$,
witness $\witR = (f, \addr)$, and the ouput oracles from $\indexer$;
the verifier $\verifier$ takes as input the instance $\instR = \bot$ and 
the indexer oracles.

\begin{lemma}
    A lemma for $\nxt{n}(\xv)$ with degree $\sps{n}$.
\end{lemma}

\textbf{Offline phase.}
The indexer outputs the oracles $t \in \bdp{n}{d_t}$, $\nxt{n} \in \bdp{n,n}{\sps{n}}$,
and $\nxt{n+1} \in \bdp{n+1,n+1}{\sps{n+1}}$.\footnote{We can also let the verifier 
compute $\nxt{n}$ and $\nxt{n+1}$ herself as the polynomials can be evaluated in $O(n)$ time.}

\textbf{Interactive phase.} 
\begin{itemize}
    \item The prover $\prover$ computes a polynomial $h \in \bdp{n+1}{1}$ ``sorted'' from $f$ and $t$.
        More precisely, let $N \deq 2^n$, define vectors
        \[
            \vec{f} \deq (f(0^n), \dots, f(\nxt{n}^{N-1}(0^n)))\,, \qquad 
            \vec{t} \deq (t(0^n), \dots, t(\nxt{n}^{N-1}(0^n))) 
        \]
        and let $\vec{h} \in \FF^{2N}$ denote the sorted version of $(\vec{f}, \vec{t})$.
        The polynomial $h$ interpolates the vector $\vec{h}$ so that for all $i \in [2N]$,
        it holds that
        \[
            \vec{h}_i = h(\nxt{n+1}^{i-1}(0^{n+1})) \,.
        \]
        
    \item The prover $\prover$ sends the oracles $f \in \bdp{n}{d_f}$ and $h \in \bdp{n+1}{1}$ to $\verifier$.
    \item The verifier $\verifier$ sends random challenges $\beta, \gamma \in \FF$.
    
    \item The prover $\prover$ and the verifier $\verifier$ run (in parallel) the interactive phase 
    of two $\RPROD$'s PIOP executions:
        \begin{itemize}
            \item the $1$st execution is with instance $\instR = (n+1, \sps{n+1}, s)$ and
                witness $\witR = \hat{h} \in \bdp{n+1}{\sps{n+1}}$, where 
                \[
                    \hat{h}(\xv) \deq \gamma(1+\beta) + h(\xv) + \beta \cdot h(\nxt{n+1}(\xv)) \,,
                \] and $s = \prod_{\xv\in\bcube{n+1}} \hat{h}(\xv)$;
                
            \item let $d \deq \max(d_f, d_t \sps{n+1})$, the $2$nd execution is with instance 
                $\instR = (n+1, d, s)$ and 
                witness $\witR = g \in \bdp{n+1}{d}$, where $g$ is defined to be 
                \[
                    g(X_1, \xv) \deq (1-X_1) \cdot \left[(1+\beta)(\gamma + f(\xv))\right]
                                    + X_1 \cdot \left[\gamma(1+\beta) + t(\xv) + \beta \cdot t(\nxt{n}(\xv))\right]\,.
                \]
                We note that for all $\xv \in \bcube{n}$, it holds that
                \[
                    g(0, \xv) \deq (1+\beta)(\gamma + f(\xv)), \qquad 
                    g(1, \xv) \deq \gamma(1+\beta) + t(\xv) + \beta \cdot t(\nxt{n}(\xv)) \,.
                \]
        \end{itemize}
\end{itemize}

\textbf{Query phase.}
The verifier $\verifier$ runs (in parallel) the query phase of the two $\RPROD$ PIOP 
executions specified above.



\section{Hyperplonk}
\begin{remark}
    TODO: batching multiple $\RZERO$ instances and multiple $\RPROD$ instances.
\end{remark} 

\clearpage
\bibliographystyle{alpha}
\bibliography{refs}

\end{document}
